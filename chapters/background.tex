\TODO

\begin{itemize}
    \item EHW
    \item POE
\end{itemize}

\section{Artificial Evolution}

\TODO

\begin{itemize}
    \item Antenna design
    \item Robot controllers
\end{itemize}

\subsection{Genetic Algoritmhs}

\TODO

\begin{itemize}
    \item Select, Cross, Mutate
    \item Genotype to Phenotype
\end{itemize}

\subsection{Evolution in Materio}

\TODO

\section{Artificial Development}

\TODO

\begin{itemize}
    \item Attractors
    \item Adaptation
    \item Graceful degradation
\end{itemize}

\subsection{Lindenmayer Systems}

\TODO

\section{Cellular Automata}

\TODO

\begin{itemize}
    \item 1D, 2D, 3D
    \item Neighborhoods
    \item Wolfram's classes
\end{itemize}

\section{FPGA}

A Field Programmable Gate Array (FPGA) is a type of reconfigurable hardware.
It can implement any desired logical operation by configuring and connecting a number of look-up tables (LUTs) and flip-flops (FFs).
FPGAs can also contain dedicated blocks for addition, multiplication, memory, and other functions.
These elements are grouped into configurable logic blocks (CLBs), which through a network of interconnects can be connected to each other or input/output pins.
An example of this structure is shown in \figurename~\ref{fig:fpga}.
Note that modern FPGAs consists of thousands of CLBs and hundreds of I/O pins \cite{ds160}.

\begin{figure}[!ht]
    \centering
    \includegraphics[width=0.60\textwidth]{fpga}
    \caption[FPGA]{
        High-level block diagram of an FPGA.
        An array of configurable logic blocks (CLBs) and input/output blocks (IOBs) are connected by a network of interconnects.
    }
    \label{fig:fpga}
\end{figure}

FPGAs have been the subject of EHW research due to their reconfigurability, and several researchers have been successful in evolving working electronic circuits \cite{huelsbergen1998evolution} \cite{thompson1997evolved}.
However, the resulting circuits have often ended up using intrinsic properties of the silicon and been very sensitive to environmental changes.

The trouble with using modern FPGAs for EHW research is that some configuration bitstrings can destroy the FPGA \cite{ug380} \cite{xapp151}.
This means that the bitstrings can not be used directly as the genotype without complicated tests to discard those that are dangerous.

\section{PCI Express}

The PCI Express interface was designed to tackle the arising trouble with clocked parallel buses like PCI.
The problem with such buses is that the clock speed can not be increased beyond a given threshold, as the slightly different lengths of the wires causes data to arrive at slightly different times.
Reducing the clock period to less than the variation in arrival time means the data will become corrupted.
This problem is exacerbated with increasing bus size.

PCI Express is therefore based on serial communication over differential pairs (lanes\footnotemark) without the need for a reference clock \cite{pcie}.
\footnotetext{
    PCI Express operates in full duplex mode, which means that each lane has an independent differential pair in each direction.
    1, 2, 4, 8, 16 or 32 lanes are supported, but data is striped and thus still transmitted serially.
}
This allows an extremely fast clock speed compared to a parallel bus, and much greater bandwidth in total.
PCI Express consists of three layers; the physical layer, the data link layer and the transaction layer, structured as shown in \figurename~\ref{fig:pcie}.

\begin{figure}[!ht]
    \centering
    \includegraphics[width=0.5\textwidth]{pcie}
    \caption[PCI Express structure]{
        High-level diagram showing the layered structure of PCI Express. (Reprinted from \cite{pcie})
    }
    \label{fig:pcie}
\end{figure}

The transaction layer's primary responsibility is the creation and parsing of transaction layer packets (TLPs).
TLPs are used to trigger events or start various transactions, most commonly to initiate read and write requests\footnotemark.
\footnotetext{
    Read and write requests are directed at one of up to six base address registers (BARs).
    They represent internal memory areas that can be anywhere from a few bytes to several gigabytes in size.
}
Most requests entail the return of a completion TLP containing the requested data or other information.
TLPs consists of multiple 32-bit double words (DW), where the first is a common header describing the type of packet.

The data link layer ensures integrity by adding error detection codes to outgoing TLPs and performing error detection and correction on incoming TLPs.
It is also responsible for retransmission if corruption occurs.

The physical layer is responsible for serialization and deserialization of the data stream.
Each byte is padded with two extra bits (8b/10b encoding) to allow clock recovery.

\section{Related Work}

\TODO

\subsection{CAM-Brain Machine}

\TODO

\subsection{CAM-8}

\begin{itemize}
    \item Cellular Automata Machine
    \item Semi-parallel CA simulator
    \item Indefinitely scalable 3D mesh-network multiprocessor
    \item Optimized for large inexpensive simulations
    \item Prototype faster than existing supercomputers
    \item Hope to whet the apetite of researchers with computing power
    \item Time-sharing of com resources interprocessor wires, allows scalability to be achieved with current tech
    \item Node: DRAM with millions of cells, SRAM look-up table
    \item Nodes update cells seqentially
    \item 3G site updates per second
    \item No traditional CA neighborhoods: Based on partitioning from lattice gas models (shifts bits around)
    \item Good at spatially moving data and interaction at lattice sites: Well suited for simulating physical systems using lattice-gas-like dynamics or localized interactions.
    \item Goal to bring computation closer to physics to improve computation
    \item Harness the astronomical computing power that is available in CA format
\end{itemize}
