Nature has many attractive properties that engineers hope to incorporate into man-made technology to revolutionize computing.
Properties include, among other things, reproduction, learning, adaption and massive parallelism.
One bio-inspired computational structure is the Cellular Automaton (CA), which mimic the massively parallel, distributed and locally interactive nature of multi-cellular organisms.

At NTNU, three master theses have gone into creating an Field-Programmable Gate Array (FPGA) platform whose goal is to allow research on CAs in combination with artificial evolution and development.
The main principle is that a Genetic Algorithm (GA) is used to create development rules, which are used to create a CA structure, which can finally be executed to compute a fitness value.
The fitness value is then fed back into the GA and the process repeated until a good solution is found.

In expectation of new hardware with a larger FPGA, the purpose of most recent thesis was to take advantage of increased resources to improve speed and to extend the CA into 3D.
However, the hardware failed to be delivered on time due to manufacturing problems.
This caused the new communication module to remain unimplemented and only allowed rudimentary simulation testing.

In a specialization project leading up to this thesis, a new communication module was implemented and integrated into the platform, allowing proper hardware verification.
It showed that the design had many issues, including failing instructions, extensive use of outdated features and separate versions for 2D and 3D CAs.
In this thesis, the entire platform has therefore been revised and rebuilt.

The new platform solves all major issues with the previous and adds further enhancements like more advanced control flow and an adaptive software API.
Higher and more fine-tunable build parameters allow wider adjustment of performance and make it possible to fit larger CAs within the FPGA.

The functionality of all instructions is verified in hardware and a program that creates replicating structures is demonstrated, proving that the platform is complete and can immediately be taken into use for research.
A clocking issue with the communication module is currently reducing the entire platform to half speed, but if fixed, the raw CA performance will be 35\% higher in 2D and 300\% higher in 3D compared to the previous design.
For those performance configurations, resource usage is significantly lower in 2D and equivalent in 3D.
