Naturen har mange attraktive egenskaper som ingeniører håper på å en dag kunne gjenskape i menneskeskapt teknologi for å revolusjonere databehandling.
De inkluderer blant annet reproduksjon, læring, tilpasning og massiv parallellisering.
Én bio-inspirert struktur er den Cellulær Automaten (CA).
Den kan gjenskape flercellede organismers massive parallellitet, distribusjon og lokal samhandling.

På NTNU har tre masteroppgaver gått med til å lage en FPGA-platform med formål om å muliggjøre forskning på CAer i kombinasjon med kunstig evolusjon og utvikling.
Hovedprinsippet er at en genetisk algoritme (GA) brukes til å lage utviklingsregler, som så blir brukt til å konstruere en CA, som til slutt brukes til å beregne en fitnessverdi.
Fitnessverdien blir så matet tilbake inn i GAen og prosessen gjentas til en god løsning er funnet.

I forventning om å få ny maskinvare med en større FPGA, gikk den nyligste masteroppgaven ut på å utnytte den økte ressursmengden til å bedre ytelsen og utvide CAen til 3D.
Men på grunn av produksjonsproblemer kunne dessverre ikke maskinvaren leveres i tide.
Derfor ble ikke en ny kommunikasjonsmodul implementert og kun grunnleggende simuleringstester kunne utføres.

I det innledende spesialiseringsprosjekt til denne masteroppgaven ble en ny kommunikasjonsmodul implementert og integrert i platformen, som tillot skikkelig maskinvareverifisering.
Den viste at designet hadde mange problemer, inkludert feilende instruksjoner, stor bruk av utdaterte elementer og separate versjone for 2D og 3D CAer.
I denne masteroppgaven har derfor hele patformen blitt revidert og ombygd.

Den nye platformen løser alle store problemer med den gamle og legger til nye forbedringer som mer avansert kontrollflyt og et adaptivt programvare API.
Flere og mer fininstillbare byggparametre tillater større justering av ytelse og gjør det mulig å få plass til større CAer på FPGAen.

Funksjonaliteten til alle instruksjoner er verifisert i maskinvare og demonstrert med et program som lager repliserende strukturer.
Dette viser at plattformen er komplett og klar for bruk til forskning.
Et klokkeproblem med kommunikasjonsmodulen fører for tiden til at hele platformen kjører på halv hastighet, men viss det fikses vil den rå CA ytelsen bli 35\% bedre i 2D og 300\% bedre i 3D sammenlignet med det forrige designet.
For disse ytelseskonfigurasjonene er ressursbruken vesentlig mindre i 2D og omtrent lik i 3D.
