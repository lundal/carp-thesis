\TODO
Unconventional machines – what are they?
Hardware implementations are much faster than simulations.
FPGAs are not as fast as ASICs, but a FPGA design is simple to port to ASIC.
CARP is a proof-of-principle.

Living organisms exhibit a range of desirable characteristics:
Graceful degradation, Robustness, Redundancy, self-assembly.

\todo{Motivation}
Research tool.
Self-assembly.
Performance/energy gain.

\todo{Method}
Bio inspired.
Evolution and development.

\todo{Conseptual description of machine}
CA + Development + Host API

\todo{Implementation in FPGA}

The initial work was done by Dupdal, and was promptly extended by Aamodt.

%==============================================================================%

\section{Outline}

The thesis is organized as follows:
\todo{more?}

\begin{itemize}
    \item Chapter \ref{ch:background} –
        Theoretical background, technology and related work.
        This chapter gives an overview of the relevant research this thesis is based on and the technology which is used.
    \item Chapter \ref{ch:previous-work} –
        Previous designs and implementations.
        This chapter states a brief history of the platform that this thesis builds on.
        All three previous master theses are covered.
    \item Chapter \ref{ch:development-platform} –
        Development platform and setup.
        This chapter describes the hardware and software systems used in this thesis and their setups.
    \item Chapter \ref{ch:implementation} –
        Implementation details.
        This chapter provides in-depth descriptions of all parts of the rebuilt and extended platform.
    \item Chapter \ref{ch:verification} –
        System verification.
        This chapter asserts the functionality of the platform through unit tests, performance benchmarks and example programs.
    \item Chapter \ref{ch:discussion} –
        Challenges and future work.
        This chapter discuss difficulties and compromises during development in addition to possible further improvements of the platform.
    \item Chapter \ref{ch:conclusion} –
        Concluding remarks.
        The final chapter concludes this paper by reviewing the new platform, its performance and its potential for future applications.
    \item Appendix \ref{app:test-descriptions} –
        Unit tests.
        This appendix briefly describes the tests programs that together provide test coverage of all instructions.
    \item Appendix \ref{app:attached-files} –
        Attachment index.
        This appendix lists the attached files comprising the hardware design and software API.
    \item Appendix \ref{app:isa} –
        Instruction Set Architecture.
        This appendix provides a complete specification of the rule format, the LUT format and all instructions.
    \item Appendix \ref{app:specialization-project} –
        Specialization project.
        This appendix holds the paper that led up to this thesis, in which the implementation and integration of the PCI Express-based communication module into the existing design is described.
\end{itemize}
