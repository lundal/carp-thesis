\TODO

The initial work was done by Dupdal, and was promptly extended by Aamodt.

\section{Motivation}

\todo{maybe not separate section}

\begin{itemize}
    \item Fix all instructions
    \item More modular and maintainable
    \item More configurable (not only powers of two)
    \item Better resource utilization
    \item Adaptive API
    \item Simpler build system
    \item More advanced control flow
    \item Toggle matrix wrapping
    \item Remove outdated code (tristates and reset)
    \item Data structures instead of prints in API
\end{itemize}

\section{Outline}

\TODO
The thesis is organized as follows:

\begin{itemize}
    \item Chapter \ref{ch:background} –
        Theoretical background, technology and related work.
        This chapter gives an overview of the relevant research this thesis is based on and the technology which is used.
    \item Chapter \ref{ch:previous-work} –
        \TODO
    \item Chapter \ref{ch:development-platform} –
        \TODO
    \item Chapter \ref{ch:implementation} –
        Implementation details.
        This chapter provides in-depth details of all parts of the rebuilt and extended platform.
        \TODO
    \item Chapter \ref{ch:verification} –
        \TODO
    \item Chapter \ref{ch:discussion} –
        \TODO
    \item Chapter \ref{ch:conclusion} –
        Concluding remarks.
        The final chapter concludes this paper by reviewing the new platform, its performance and its potential for future applications.
    \item Appendices –
        \TODO
\end{itemize}
