\TODO
How it was to implement a CA system inside an FPGA.
Distributed and local communication allows it to fills entire FPGA.
Perfect for 2D structure, extra routing challenges with 3D.

\begin{itemize}
    \item Compromise between generics and custom types (VHDL-2008 not supported)
\end{itemize}

\TODO
Many resources left.
Scaling restricted by routing; almost all paths are critical.
CA shift registers are spread over whole FPGA.
It is summed into one bram (livecount).
Then spread all over, since DFT uses nearly all DSP slices.

\section{Warnings}

\begin{itemize}
    \item Most are due to parameterization
    \item Many from generated PCIE core
    \item Nearly all filtered to make output more readable and make it easier to find bugs
    \item Could not remove remaining due to limited filter system (don't want to cause trouble later)
    \item 728 in ISE (684 filtered)
    \item 502 in XST (457 filtered)
    \item (3D)
    \item Why the difference?
    \item Also filtered "equivalent register removal" infos to reduce clutter
\end{itemize}

\section{Challenges}
\label{sec:challenges}

\TODO

\section{Future work}

\TODO

Further parameterization:
Could parameterize number of rows accessed in Cell Storage to allow higher config and readback speeds.
Also live counter to use more cycles but reduce LUT usage.
However, LUT and register usage is not an issue; routing and available shift registers are.
Shift registers set max number of cells.
Routing limits livecounter + rule testers.

\begin{itemize}
    \item Hazard detection unit so multiple modules can run simultaneously
    \item DMA-based communication interface (better speed, requires driver, more advanced com module)
    \item Watchdog-timer/Stall-prevention
\end{itemize}
