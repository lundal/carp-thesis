In this paper, NTNU's CA research platform has been completely re-engineered.
The overall structure is equivalent, but the hardware design has been made more modular, configurable and structured, and the software API more adaptable, complete and user-friendly.
The 2D and 3D designs have been unified and any external dependencies removed.

It has been thoroughly tested in hardware and solves nearly all issues with the previous design.
There are however some issues with the communication module, which is ironic as its implementation was the original task of the project.
However, when the design is restored to 125 MHz, the raw CA performance should be 35\% higher in 2D and about 300\% higher in 3D at lower or equivalent resource usage.

In essence, this thesis provides a complete tool for CA research and experimentation.
It allows study of self-organization, adaptation and... \TODO

It can be integrated into any system with a recent version of Linux and an available PCI Express socked.

\todo{remove these notes}
Complete tool for CA research.
Tried and tested (examples).
Good performance.
Should be easily extendable for larger FPGAs.
Can be integrated into current systems.
Research platform for self-org, adaptation, ++
List use cases.
Thoroughly tested.
