In this paper, NTNU's CA research platform has been completely re-engineered.
The overall structure is equivalent, but the hardware design has been made more modular, configurable and structured, and the software API more adaptable, complete and user-friendly.
The 2D and 3D designs have been unified and any external dependencies removed.
It should also be easily extendable to larger FPGAs.

The platform has been thoroughly tested in hardware and all issues with the previous design that are listed in Table~\ref{tab:issues} have been resolved.
There are however some performance setbacks with the communication module, which is slightly ironic as its implementation was the original task of the project.
However, when the design is restored to 125 MHz, the raw CA performance should be about 35\% higher in 2D and 300\% higher in 3D at lower or equivalent resource usage.

In essence, this thesis provides a complete tool for CA research and experimentation, and allows study of self-organization, adaptation, replication and other applicable biological processes.
It can be integrated into any computer system with a recent version of Linux and an available PCI Express socked.
